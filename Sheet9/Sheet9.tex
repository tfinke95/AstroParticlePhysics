\documentclass[10pt,a4paper]{article}
\usepackage[utf8]{inputenc}
\usepackage[english]{babel}
\usepackage{amsmath}
\usepackage{amsfonts}
\usepackage{amssymb}
\usepackage{graphicx}
\begin{document}
\subsection*{Exercise 1}
\paragraph{a)}
\begin{align*}
E_{kin} &= \gamma m_0 c^2 - m_0 c^2 = m_0 c^2 (\gamma - 1) \\
\frac{E_{kin}}{m_0 c^2} &= \frac{1}{\sqrt{1-\beta^2}} - 1 \\
\beta &= \sqrt{1-\left(\frac{E_{kin}}{m_0 c^2} + 1\right)^{-2}}
\end{align*}
Now rewriting the energy density as
\begin{align*}
\rho_E = \rho_{E_0} + \rho_{E_{kin}}
\end{align*}
where $\rho_{E_0}$ is the energy density occurring from the rest mass and $\rho_{E_{kin}}$ is the kinetic energy density. Analogous to the calculation in the script one gets:
\begin{align*}
\rho_E &= \rho_{E_0} + \int_{E_{min}}^\infty dE_{kin}\ \frac{dn_{CR}}{dE_{kin}}E_{kin}\\
&=\rho_{E_0} + \int_{E_{min}}^\infty dE_{kin}\ \frac{4\pi}{\beta c}\frac{d\Phi}{dE_{kin}}E_{kin}\\
&=\rho_{E_0} + \frac{4\pi \Phi_0}{c} \int_{E_{min}}^\infty dE_{kin}\ \frac{E_{kin}^{-\gamma+1}}{\sqrt{1-\left(\frac{E_{kin}}{m_0 c^2} + 1\right)^{-2}}}\\
\end{align*}
In the last step the power-law for the kinetic energy flux is inserted and the expression for $\beta$ from before.

\paragraph{b)}
From the graph one can read the points $P_1 = (10 | 2\cdot 10^{-3})$ and $P_2=(10^2 | 5 \cdot 10^{-6})$. This results in $\Phi_0 = 0.8\ (cm^2\ s\ sr\ GeV)^{-1}$ and $\gamma = 2.602$.

\paragraph{c)}
Inserting into the integral values from b) gives a value
\begin{align*}
\int_{1\ GeV}^\infty dE_{kin}\ \frac{E_{kin}^{-\gamma+1}}{\sqrt{1-\left(\frac{E_{kin}}{m_0 c^2} + 1\right)^{-2}}} = 0.9182
\end{align*}
Resulting in $\rho_{E_{kin}} = 3.0790 \cdot 10^{-10}\ \frac{GeV}{cm^3}$.

\paragraph*{d)}
C is approximately 0.1. For constant flux one finds the same integral just with $\gamma = 0$. This again results in $\rho_{E_kin} = 3.6552\ \frac{GeV}{cm^3}$.

\subsection*{Exercise 4}
\paragraph*{a)}
The energy gain per time is given by 
\begin{align*}
\frac{dE}{dt} = &Collisionrate \cdot  \delta E \\
= &\frac{\xi E}{t_{cycle}} = \frac{E}{\alpha\ t_{esc}}\\
=> E(t) = &E_0 \cdot \exp\left(\frac{\xi}{t_{cycle}}\cdot t\right) = E_0 \cdot \exp\left(\frac{t}{\alpha\ t_{esc}}\right)
\end{align*}
Here $\xi$ is the relative energy gain per acceleration. $t_{cycle}$ is the time between to cycles, $\alpha$ is the spectral index and $t_{esc}$ is the escape time.

\paragraph{b)}
The acceleration time is given by
\begin{align*}
t_{acceleration} = \frac{t_{cycle}}{\xi} \cdot ln\left(\frac{E}{E_0}\right)
\end{align*}
$\xi = \frac{4}{3} \beta_V^2 = 3.7088 \cdot 10^{-8}$, $t_{cycle}$ is limited by d to $\approx 3\ a$. So one gets
\begin{align*}
t_{acceleration} = 1.21 \cdot 10^9\ a
\end{align*}

\paragraph{c)}
\begin{align*}
\frac{E(t=6\cdot 10^6\ a)}{E_0} = \exp\left(\frac{\xi}{t_{cycle}}\cdot 6\cdot 10^6\ a\right) = 1.077
\end{align*}

\paragraph{d)}
The new energy gain is given by $\xi = 3.7088\cdot 10^{-6}$. The new cycle time $t_{cycle} \approx 3.2616 \cdot 10^6\ a$. One gets for the acceleration from $1\ GeV$ to $10^{19.5}\ eV$ a time of $2.1261 \cdot 10^{13}\ a$. \\
For the acceleration from $10^{15}\ eV$ one gets a time of $0.1122 \cdot 10^{12}\ a$.
\end{document}