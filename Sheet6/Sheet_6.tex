\documentclass[10pt,a4paper]{article}
\usepackage[utf8]{inputenc}
\usepackage[english]{babel}
\usepackage{amsmath}
\usepackage{amsfonts}
\usepackage{amssymb}
\usepackage{graphicx}
\usepackage{hyperref}
\begin{document}
\section*{Exercise 4}
\paragraph{a)}
"The isotope $^{10}Be$ predominately measures the confinement time for C, N and O." \footnote{\url{http://www.srl.caltech.edu/ACE/CRIS_SIS/cris-ssr-paper.pdf}}
\paragraph{b)}
Alternatives for $^{10}Be$ are $^{26}Al$, $^{36}Cl$, $^{14}C$ and $^{54}Mn$.
\paragraph{c)}
The lifetime of $^{14}C$ is short, compared to the others.
It is approximately 5730 years. That is why it can only be observed, if it is produced by fragmentation of heavier nuclei in the neighbourhood of the solar system.\\
For $^{54}Mn$ the constraint is that the decay time has not been measured directly.
\paragraph{d)}
\begin{align*}
^{10}Be/^{9}Be &= 0.12\ now\\
^{10}Be/^{9}Be &= 0.48\ beginning\\
\lambda_{esc} &= \frac{26.7\cdot \beta \ g\ cm^{-2}}{(\beta R / 1GV)^{0.58}+(\beta R / 1.4GV)^{-1.4}}\\
\tau_{esc} &= \lambda_{esc} / \rho_{ISM} \beta c \\
\tau_{^{10}Be} &= 2.3 \cdot 10^6 yr\\
\epsilon &= \rho_{ISM} / \rho_{CRIS} = ???
\end{align*}
\paragraph{e)}
\begin{align*}
\tau_{esc}^{phys} &= \lambda_{esc}/\rho_{CRIS} \beta c = \tau_{esc} \cdot \rho_{ISM} / \rho_{CRIS} = \tau_{esc} \cdot \epsilon
\end{align*}
For looking at the both ratios given in exercise d) and then assuming an exponential decay of $^{10}Be \rightarrow ^{10}Be(t) = ^{10}Be_0 \cdot \exp(-t/\tau)$ one can derive
\begin{align*}
0.12 &= \frac{^{10}Be \exp(-t/\tau)}{^{9}Be + (1 - \exp(-t/\tau)) ^{10}Be} \\
&= \frac{0.48 \exp(-t/\tau)}{1 + (1 - \exp(-t/\tau)) 0.48}
\end{align*}
In the last step the initial ratio $^{10}Be/^{9}Be=0.48$ is used. 
Solving for the time $t$ using $\tau=2.3 \cdot 10^6\ yr$ one gets $t=2.546 \cdot 10^6\ yr$
\paragraph{f)}
For increasing energy I would expect shorter escape times. For higher energies the bending radii from magnetic fields become larger and a particle leaving the galactic disk with large energy has a higher probability to not being bent strong enough to re-enter the disk.
\end{document}