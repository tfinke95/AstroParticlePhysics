\documentclass[10pt,a4paper]{article}
\usepackage[utf8]{inputenc}
\usepackage[english]{babel}
\usepackage{amsmath}
\usepackage{amsfonts}
\usepackage{amssymb}
\usepackage{graphicx}
\begin{document}
\subsection*{Exercise 2}
The Hillas criterion reads:
\begin{align*}
\frac{E_{max, obs}}{10^{20}\ eV} \lesssim \Gamma \cdot Z \cdot \frac{L_{source}}{pc} \cdot \frac{B}{G}
\end{align*}
For the LHC $B$ goes up to $7.7\ T = 7.7\cdot 10^4\ G$. The circumference of the LHC is $\approx 27\ km$ resulting in a diameter $L_{source} = 2.7852 \cdot 10^{-13}\ pc$. For us the LHC itself is not moving relativistically ($\Gamma = 1$) and $Z = 1$ for protons.
\begin{align*}
\Rightarrow E_{max, obs} \lesssim 2.1446\ TeV
\end{align*}
The maximum energy reached in the LHC is $6.5\ TeV$ for one proton. This can be explained by an additional factor in the Hillas criterion taking into account the acceleration efficiency. For shock acceleration a factor $\frac{\beta_S}{0.2}$ is added. A similar factor will be necessary for the acceleration efficiency of the LHC. Using the maximum for shock acceleration (factor $5$) one gets
\begin{align*}
E_{max, obs} \lesssim 10.723\ TeV.
\end{align*}

\subsection*{Exercise 4}
\paragraph{a)}
For $T = 2.725\ K$ one gets for the integral $n_\gamma = 1.9334\cdot 10^14\ m^{-3}$. The integral is taken from 0 to $10^{18}$, since computational calculation above was not possible.

\paragraph{b)}
The mean free path is given by the inverse of the product of number density and cross section:
\begin{align*}
\lambda = \frac{1}{n_\gamma \sigma_T} = \frac{1}{1.2857 \cdot 10^{-14}}\ m = 7.7778 \cdot 10^{13}\ m = 3.2408 \cdot 10^{-3}\ pc
\end{align*}

\paragraph{c)}
Since our galaxy is about $30\ kpc$ across, it seems very unlikely that the electrons could come extra-galactic sources, since their mean free path is way too short.
\end{document}